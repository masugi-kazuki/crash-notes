\documentclass{jsarticle}


\setcounter{secnumdepth}{5}
\setcounter{tocdepth}{1}

\usepackage{amsmath}
\usepackage{amssymb}
\usepackage{amsthm}
\usepackage{tikz}
\usetikzlibrary{cd}
\usepackage{enumerate}
\usepackage{mathtools}

\usepackage[dvipdfmx]{hyperref}
\usepackage{pxjahyper}
\hypersetup{% hyperref
setpagesize=false,
 bookmarksnumbered=true,%
 bookmarksopen=true,%
 colorlinks=true,%
 linkcolor=blue,
 citecolor=red,
}

\newcommand{\pr}[1]{\mathrm{pr}_{#1}}
\newcommand{\Sch}{\mathsf{Sch}}
\newcommand{\Spec}{\mathrm{Spec}}

\renewcommand{\proofname}{\textbf{証明}}
\newtheorem{bigthm}{大定理}
\newtheorem{thm}{定理}
\newtheorem{prop}[thm]{命題}
\newtheorem{lem}[thm]{補題}
\newtheorem{defn}[thm]{定義}
\newtheorem{rem}[thm]{Remark}
\usepackage[utf8]{inputenc}

\title{群スキームの基礎}
\author{馬杉和貴}
\date{\today}
\begin{document}
\maketitle

この記事において、以下の notation を固定する。

\begin{itemize}
  \item $S$ はスキームである。
  \item $G$ は $S$ 上の群スキームである。
  \item $\mu_G \colon G \times_S G \to G$ は $G$ の群構造のもとでの乗法射である。
  \item $c_G \colon G \to G$ は $G$ の群構造のもとでの逆元射である。
  \item $e_G \colon S \to G$ は $G$ の群構造のもとでの単位元射である。
  \item $\pr{i}$ という記号は、(文脈上適切な意味においての) 第 $i$-成分への射影をあらわす。
  \item $\sigma_G \colon G \times_S G \to G \times_S G$ を、$(\pr{1}, \mu_G)$ なる同型射とする。
  \item $k$ を体とする。また、$A$ という記号は、local Artin 環をあらわすものとし、$A$ という記号が文脈中あらわれるとき、$k$ は $A$ の剰余体であるものとする。
  \item スキーム $X$ について、$X$ 上スキームの圏を $\Sch_X$ と表記する。また、$X$ 上の被約なスキームのなす圏を $\Sch_{X, \mathrm{red}}$ と表記する。
\end{itemize}

\section{section 1 (Provisinal title)}

cf. SGA3, Expos\'{e} VI.

\begin{prop}
  $G$ を $S$ 上の局所有限表示平坦群スキームとする。$G$ が $S$ 上 geometrically reduced であることと、smooth であることは同値である。
\end{prop}
\begin{proof}
  $S$ を代数閉体 $k$ として議論をしてよい。smooth ならば reduced であることは明らかである。よって、逆方向の力学を確認すればよい。

  $G$ の任意の閉点 $x$ について、$G$ が $x$ で regular であることを示す。$k$ の代数閉性より、$G$ がある閉点のもとで regular であることを示せばよい。ここで、$G$ の reducedness は、generically smooth であることを導く。したがって、これは所望の主張を導く。
\end{proof}

\begin{prop}
  $G$ を $A$ 上の局所有限表示平坦群スキームとする。このとき、$G$ は Cohen-Macaulay であり、さらに、任意の点 $x \in G$ について $\mathcal{O}_{G, x}$ のパラメータ系 $a_1, \ldots, a_n$ であって $\mathcal{O}_{G, x} / (a_1, \ldots, a_n)$ が $A$ 上有限自由加群となるようなものが存在する。
\end{prop}
\begin{proof}
  これらの主張は、$A$ がその剰余体である場合に帰着できる。また、体 $k$ 上本質的に有限型の環 $A$ に対して、その Cohen-Macaulay 性は、$k$ の有限拡大体 $K$ に対する $A \otimes_k K$ の Cohen-Macaulay 性と同値である。このことに着目すれば、平行移動に関するトリックを用いて、ある閉点に対してその Cohen-Macaulay 性を確認すればよい。しかしこのことは、次の一般的なスキーム論の補題により確認される。
\end{proof}

\begin{lem}
  体 $k$ 上の非空な局所有限型スキーム $X$ について、ある閉点 $x \in X$ が存在して、$\mathcal{O}_{X, x}$ は Cohen-Macaulay である。
\end{lem}
\begin{proof}
  $X$ を affine 開集合に取り替えてよい。$X = \Spec(B)$ として、$B$ の次元に関する induction をまわす。$\dim B = 0$ のときは明らかである。

  $\dim B > 0$ のとき、$B$ の非単元であって、かつ非零因子であるような元が存在するため (仮定を用いている)、そのような元 $f \in B$ をとる。$B / (f)$ の次元は $B$ の次元よりも小さいので、帰納法の仮定により、ある閉点 $y \in \Spec(B / (f))$ が存在して、$\mathcal{O}_{\Spec(B / (f)), y}$ は Cohen-Macaulay である。ここで、$y$ を $X$ に引き戻すと、$\mathcal{O}_{X, y}$ は Cohen-Macaulay であることがわかる。
\end{proof}






















\end{document}





















