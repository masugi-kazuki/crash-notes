\documentclass{jsarticle}


\setcounter{secnumdepth}{5}
\setcounter{tocdepth}{1}

\usepackage{amsmath}
\usepackage{amssymb}
\usepackage{amsthm}
\usepackage{tikz}
\usetikzlibrary{cd}
\usepackage{enumerate}
\usepackage{mathtools}

\usepackage[dvipdfmx]{hyperref}
\usepackage{pxjahyper}
\hypersetup{% hyperref
setpagesize=false,
 bookmarksnumbered=true,%
 bookmarksopen=true,%
 colorlinks=true,%
 linkcolor=blue,
 citecolor=red,
}

\newcommand{\Pic}{\mathrm{Pic}}
\newcommand{\Sch}{\mathsf{Sch}}

\newcommand{\Gm}{\mathbb{G}_m}
\newcommand{\Ga}{\mathbb{G}_a}

\renewcommand{\H}{\mathrm{H}}
\renewcommand{\O}{\mathcal{O}}
\newcommand{\R}{\mathbb{R}}


\renewcommand{\proofname}{\textbf{証明}}
\newtheorem{bigthm}{大定理}
\newtheorem{thm}{定理}
\newtheorem{prop}[thm]{命題}
\newtheorem{lem}[thm]{補題}
\newtheorem{defn}[thm]{定義}
\newtheorem{rem}[thm]{Remark}
\usepackage[utf8]{inputenc}



\title{AlgGeomBasic}
\author{馬杉和貴}
\date{\today}
\begin{document}
\maketitle

このノートは、著者の研究活動上のメモ・備忘録としての側面をもつものである。著者は (執筆時点においては) 遠アーベル幾何学を専攻する修士課程の学生である。したがって、内容については、遠アーベル幾何学を習得するにあたって必要となった、代数幾何学に関する (一般的な) 基礎事項をまとめたものとなる。

\tableofcontents

\section{algebraic space}

\newpage
\section{Picard scheme}

scheme $X$ について、$\Pic(X)$ は $X$ 上の line bundle の同型類全体のなすアーベル群として定められる。ここで、自然に $\Pic(X) \cong \H^1(X, \O_X^\times)$ なる同型が与えられる。

scheme $S$ を以下固定する。$S$-scheme $X$ について、次の方法で $\Sch_S$ 上の前層を与えることができる:
\begin{itemize}
  \item $S$-scheme $T$ に対して、$\Pic(X \times_S T)$ を充てる。
\end{itemize}

この方法によって構成される関手を、$X / S$ に関する \textbf{naive relative Picard functor}(素朴 Picard 関手)と呼ぶ。また、naive relative Picard functor を fppf-位相のもとで層化した関手を $\Pic_{X / S}$ と表記し、これを \textbf{relative Picard functor}(相対 Picard 関手)と呼ぶ。$S$-scheme としての構造射を $\pi \colon X \to S$ とおくと、明らかに、$\Pic_{X / S}$ は $\R^1\pi_*(\Gm)$ として表示される (ここで、pushforward は fppf-位相のもとで計算している)。

\begin{lem}
  
\end{lem}

\newpage
\section{group scheme}

\section{Albanese variety}

\section{N\'{e}ron model}

\newpage
\section{Appendix: some glimpses of scheme theory}
\begin{prop}[Noetherian approximation]
  次が成り立つ:
  \begin{enumerate}
    \item $S$ を qcqs scheme とする。このとき、ある scheme の逆系 $\{S_i, p_{i, i'}\}_{i, i'\in I}$ が存在して、以下の条件を充たす。
      \begin{itemize}
        \item 構造射 $p_{i, i'}$ は affine である。
        \item $S_i$ は $\mathbb{Z}$ 上有限型である。
        \item $S = \lim S_i$ が成り立つ。
      \end{itemize}
    \item $f \colon X \to S$ を qcqs morphism とする。このとき、ある scheme の射の逆系 $\{f_i \colon X_i \to S_i, p_{X, i, i'}, p_{S, i, i'}\}$ が存在して、以下の条件を充たす。
      \begin{itemize}
        \item 構造射 $p_{X, i, i'}$, $p_{S, i, i'}$ は affine である。
        \item $X_i$, $S_i$ は $\mathbb{Z}$ 上有限型である。
        \item $X \to S = \lim X_i \to S_i$ が成り立つ。
      \end{itemize}
    \item $f \colon X \to S$
  \end{enumerate}
\end{prop}

\begin{prop}[Stein 分解]
  $f \colon X \to S$ を proper, of finite presentation
\end{prop}




\end{document}