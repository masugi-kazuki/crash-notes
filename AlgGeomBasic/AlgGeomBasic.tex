\documentclass{jsarticle}


\setcounter{secnumdepth}{5}
\setcounter{tocdepth}{1}

\usepackage{amsmath}
\usepackage{amssymb}
\usepackage{amsthm}
\usepackage{tikz}
\usetikzlibrary{cd}
\usepackage{enumerate}
\usepackage{mathtools}

\usepackage[dvipdfmx]{hyperref}
\usepackage{pxjahyper}
\hypersetup{% hyperref
setpagesize=false,
 bookmarksnumbered=true,%
 bookmarksopen=true,%
 colorlinks=true,%
 linkcolor=blue,
 citecolor=red,
}

\newcommand{\Pic}{\mathrm{Pic}}
\newcommand{\Sch}{\mathsf{Sch}}

\newcommand{\Gm}{\mathbb{G}_m}
\newcommand{\Ga}{\mathbb{G}_a}

\renewcommand{\H}{\mathrm{H}}
\renewcommand{\O}{\mathcal{O}}
\newcommand{\R}{\mathbb{R}}


\renewcommand{\proofname}{\textbf{証明}}
\newtheorem{bigthm}{大定理}
\newtheorem{thm}{定理}[section]
\newtheorem{prop}[thm]{命題}
\newtheorem{lem}[thm]{補題}
\newtheorem{defn}[thm]{定義}
\newtheorem{rem}[thm]{Remark}
\newtheorem{note}[thm]{補足}
\usepackage[utf8]{inputenc}



\title{AlgGeomBasic}
\author{馬杉和貴}
\date{\today}
\begin{document}
\maketitle

このノートは、著者の研究活動上のメモ・備忘録としての側面をもつものである。著者は (執筆時点においては) 遠アーベル幾何学を専攻する修士課程の学生である。したがって、内容については、遠アーベル幾何学を習得するにあたって必要となった、代数幾何学に関する (一般的な) 基礎事項をまとめたものとなる。また、スキーム論についての記述がいくらか大雑把なものとなることがあるが、これは内容の不正確性あるいは informality をいうものではない (同時に、一応 Appendix としてスキーム論の記述を設けているが、必ずしも参照をおこなうとは限らず、むしろことわりなく内容を用いることが多い)。あるいは本記事の読みやすさ・理解しやすさについては、(もちろん目的に付随して一定の配慮はめざすところではあるものの) 保証するものではない。

\tableofcontents
\newpage
\section{algebraic space}

\newpage
\section{group scheme}

この section においては、群スキームに関する基本的な事実とその証明を列挙的にまとめる。内容構成は、したがって、特に非体系的である。

以下、次の notation を以下固定し、特別の言及のない限りは、文脈はこの notation に従うものとする:
\begin{itemize}
  \item $S$ は scheme とする。
  \item $G$ は $S$-group scheme とする。
  \item $\mu_G$ (あるいは文脈上明らかな場合は $\mu$) は $G$ の乗法射を表す。
  \item $1_G$ (あるいは文脈上明らかな場合は $1$) は $G$ の単位元を表す。
  \item $i_G$ (あるいは文脈上明らかな場合は $i$) は $G$ の逆射を表す。
  \item $k$ は体とする。
  \item $A$, $R$ は環であるとする。
\end{itemize}

\subsection{base 上 geometrically reduced な平坦群スキームは smooth である}
base 上 geometrically reduced な平坦群スキーム $G / S$ が smooth であることをみる。このためには、(flatness より) fiberwise に確認すればよいため、base $S$ を代数閉体 $k$ としてよい。このとき、reducedness の仮定より、$G$ は generically smooth である。このとき、適当な閉点のもとでの平行移動をおこなうことによって、$G$ が smooth であることが理解される。

\subsection{標数 $0$ のスキーム上の平坦群スキームは smooth である}
さきと同様の帰着の方法によって、標数 $0$ の代数閉体 $k$ 上で議論をおこなえばよい。

群スキーム $G / k$ について、その微分加群 $\Omega_{G / k}$ は locally free であるから、これは $G$ の smoothness をみちびく (これは、Stacks Project, Tag 04QN などを参照されたい)。

\subsection{アーベルスキームは可換群スキームである}
\begin{defn}
  アーベルスキームとは、群スキームであって、かつ smooth, proper, geometrically connected なものをいう。
\end{defn}

\begin{prop}
  アーベルスキーム $A \to S$ は、可換群スキームである。
\end{prop}
\begin{proof}
  smooth かつ proper ということから、Noetherian approximation の技術を用いて、$S$ を Noetherian の場合に帰着させることができる。これは、次の rigidity lemma によって、$S$ が体の場合に帰着される。しかしこの場合はアーベル多様体の一般論から従う。
\end{proof}

\begin{lem}
  $S$ を connected scheme とする。$f \colon X \to Y$ なる $S$-スキームの射について、$X$ が $S$ 上 proper, flat であるとする。また、任意の $s \in S$ について、$\H^0(X_s, \mathcal{O}_{X_s}) = \res(s)$ であったとする。このとき、ある一点 $s_0 \in S$ において $f(X_s)$ が set-theoretical に一点であるならば、ある $Y \to S$ の section が存在して、$f$ はその section によって定義される。
\end{lem}
\begin{proof}
  $X \to S$ なる構造射を $p$ とよぶ。descent theory の援用によって、$p$ が section $t \colon S \to X$ をもつと仮定してよい。このとき、$f \circ t \circ p : X \to S \to X \to Y$ なる方法で定義される射と、$f$ との比較をおこなう。

  $S$ が一点の場合に示されたとして、以下議論を進める。$Z \subset X$ を、$f = f \circ t \circ p$ を充たす $X$ の閉部分集合とする。前提は、$s_0$ に集中する $S$ 上の Artin 閉部分スキーム $T$ について、$Z$ が $p^{-1}(T)$ を含むことを意味する。したがって、(簡単なスキーム論的考察から、) $Z$ が $X_{s_0}$ の開近傍を含むことが理解される。$p$ の properness から、これはある $s_0$ の近傍 $U$ において、$Z$ が $p^{-1}(U)$ を含むことを意味する。この議論によって、$s \in S$ であって $X_s \subset Z$ を充たすもの全体の集合を $V$ とおくと、$V$ は open であることが理解される。しかし、$p$ の flatness より、$V$ の補集合もまた open であると理解される。したがって、$S$ の連結性によって所望の結論を得る。

  よって、$S$ が Artin local である場合に帰着された。section の存在を示すために、実際に section を構成する。位相空間の射としては、あたりまえのものを充てればよい。また、$t \colon S \to Y$ なる topological section について、$\mathcal{O}_Y \to t_*(\mathcal{O}_S)$ なる層の射には、$\mathcal{O}_Y \to f_*(\mathcal{O}_X) = t_*(p_*(\mathcal{O}_X)) = t_*(\mathcal{O}_S)$ として得られるものを充てる。この方法によって得られた環付き空間の射は、スキームの射となっており、また所望の条件を充たす。
\end{proof}

\subsection{連結成分}
\begin{prop}
  $k$ を体、$S$ を $\Spec k$ とする。$G$ を connected $S$-group scheme とする。このとき、$G$ は geometrically connected である。
\end{prop}
\begin{proof}
  $G$ が $S$ 上の rational point をもつ (単位元 $1_G$ は rational point である ( ! )) ことに注意すれば、scheme 論の簡単な exercise である。
\end{proof}

\begin{lem}
  $k$ を体、$S$ を $\Spec k$ とする。$G$ を $S$ 上 locally of finite type な $S$-group scheme とする。このとき、$G$ の単位元を含む連結成分は clopen subgroup となる。
\end{lem}
\begin{proof}
  $G$ は locally Noetherian, 特に locally connected であるため、任意の連結成分は open である。したがって、補題は明らかである。
\end{proof}

\begin{lem}
  $k$ を体、$S$ を $\Spec k$ とする。$G$ の単位元を含む連結成分は closed subgroup となる。
\end{lem}
\begin{proof}
  明らかである。
\end{proof}

\begin{prop}
  $k$ を体、$S$ を $\Spec k$ とする。$G$ を $S$ 上 locally of finite type な $S$-group scheme とする。このとき、$G$ の連結成分はいずれも geometrically irreducible, of finite type であり、かつすべておなじ次元をもつ。
\end{prop}
\begin{proof}
  $G$ の単位元をふくむ連結成分を $G_0$ とよぶ。このとき、命題のすべての主張は、$G_0$ の場合に帰着される。したがって、以下 $G$ を connected として議論をおこなう。

  $G$ が irreducible であることをみるためには、$k$ を代数閉体としてよい。このとき、$G$ の被約化 $G_\mathrm{red}$ もまた $k$ 上の群スキームとなるが、これは smooth である。よって、$G$ の下部位相空間は $G_\mathrm{red}$ の下部位相空間と標準的に同相であるから、これは $G$ の irreducible をみちびく。

  $G$ の of finite type 性のためには、$G$ が quasi-compact であることをみればよいが、これは次の scheme 論的な補題によって確認できる。
\end{proof}

\begin{lem}
  $k$ を体、$S$ を $\Spec k$ とする。$G$ を $S$-group scheme とする。このとき、$U$, $V$ が $G$ の dense open subset であるならば、$\mu|_{U \times_S V} \colon U \times_S V \to G$ は surjective である。
\end{lem}
\begin{proof}
  (体拡大が universally open であること、あるいは平行移動に関するトリックなどをおもいだせば、) $\mu|_{U \times_S V}$ の像が単位元 $1$ を含んでいることを確認すればよいことが理解される。しかし、$U$, $V$ が dense open subset であることから、特に $U \cap V^{-1}$ は非空な交差をもつ。したがって、これは補題を示す。
\end{proof}

\subsection{Cohen-Macaulay 性}
\begin{prop}
  $G$ を $A$ 上の局所有限表示平坦群スキームとする。このとき、$G$ は Cohen-Macaulay であり、さらに、任意の点 $x \in G$ について $\mathcal{O}_{G, x}$ のパラメータ系 $a_1, \ldots, a_n$ であって $\O_{G, x} / (a_1, \ldots, a_n)$ が $A$ 上有限自由加群となるようなものが存在する。
\end{prop}
\begin{proof}
  これらの主張は、$A$ がその剰余体である場合に帰着できる。また、体 $k$ 上本質的に有限型の環 $A$ に対して、その Cohen-Macaulay 性は、$k$ の有限拡大体 $K$ に対する $A \otimes_k K$ の Cohen-Macaulay 性と同値である。このことに着目すれば、平行移動に関するトリックを用いて、ある閉点に対してその Cohen-Macaulay 性を確認すればよい。しかしこのことは、次の一般的なスキーム論の補題により確認される。
\end{proof}

\begin{lem}
  体 $k$ 上の非空な局所有限型スキーム $X$ について、ある閉点 $x \in X$ が存在して、$\O_{X, x}$ は Cohen-Macaulay である。
\end{lem}
\begin{proof}
  $X$ を affine 開集合に取り替えてよい。$X = \Spec(B)$ として、$B$ の次元に関する induction をまわす。$\dim B = 0$ のときは明らかである。

  $\dim B > 0$ のとき、$B$ の非単元であって、かつ非零因子であるような元が存在するため (仮定を用いている)、そのような元 $f \in B$ をとる。$B / (f)$ の次元は $B$ の次元よりも小さいので、帰納法の仮定により、ある閉点 $y \in \Spec(B / (f))$ が存在して、$\O_{\Spec(B / (f)), y}$ は Cohen-Macaulay である。ここで、$y$ を $X$ に引き戻すと、$\O_{X, y}$ は Cohen-Macaulay であることがわかる。
\end{proof}

\subsection{次元等式}

\subsection{群スキームに関するスキーム論的補題}


\subsection{商構成}


\subsection{可換群スキームのなす圏はアーベル圏である}

\subsection{Chevalley の定理}

\subsection{semi-abelian scheme の subquotient は semi-abelian である}

\subsection{semi-abelian scheme の Galois covering は semi-abelian scheme である}



\newpage
\section{Picard scheme}

scheme $X$ について、$\Pic(X)$ は $X$ 上の line bundle の同型類全体のなすアーベル群として定められる。ここで、自然に $\Pic(X) \cong \H^1(X, \O_X^\times)$ なる同型が与えられる。

scheme $S$ を以下固定する。$S$-scheme $X$ について、次の方法で $\Sch_S$ 上の前層を与えることができる:
\begin{itemize}
  \item $S$-scheme $T$ に対して、$\Pic(X \times_S T)$ を充てる。
\end{itemize}

この方法によって構成される関手を、$X / S$ に関する \textbf{naive relative Picard functor}(素朴 Picard 関手)と呼ぶ。また、naive relative Picard functor を fppf-位相のもとで層化した関手を $\Pic_{X / S}$ と表記し、これを \textbf{relative Picard functor}(相対 Picard 関手)と呼ぶ。$S$-scheme としての構造射を $\pi \colon X \to S$ とおくと、明らかに、$\Pic_{X / S}$ は $\R^1\pi_*(\Gm)$ として表示される (ここで、pushforward は fppf-位相のもとで計算している)。

\begin{lem}
  
\end{lem}

\newpage
\section{Jacobian scheme}


\newpage
\section{Albanese variety}

\newpage
\section{N\'{e}ron model}

\newpage
\section{Appendix: some glimpses of scheme theory}
\begin{prop}[Noetherian approximation]
  次が成り立つ:
  \begin{enumerate}
    \item $S$ を qcqs scheme とする。このとき、ある scheme の逆系 $\{S_i, p_{i, i'}\}_{i, i'\in I}$ が存在して、以下の条件を充たす。
      \begin{itemize}
        \item 構造射 $p_{i, i'}$ は affine である。
        \item $S_i$ は $\mathbb{Z}$ 上有限型である。
        \item $S = \lim S_i$ が成り立つ。
      \end{itemize}
    \item $f \colon X \to S$ を qcqs morphism とする。このとき、ある scheme の射の逆系 $\{f_i \colon X_i \to S_i, p_{X, i, i'}, p_{S, i, i'}\}$ が存在して、以下の条件を充たす。
      \begin{itemize}
        \item 構造射 $p_{X, i, i'}$, $p_{S, i, i'}$ は affine である。
        \item $X_i$, $S_i$ は $\mathbb{Z}$ 上有限型である。
        \item $X \to S = \lim X_i \to S_i$ が成り立つ。
      \end{itemize}
    \item $f \colon X \to S$
  \end{enumerate}
\end{prop}

\begin{prop}[Stein 分解]
  $f \colon X \to S$ を proper, of finite presentation
\end{prop}

\newpage{Appendix: algebraic stack}




\end{document}